
\Lecture{Jayalal Sarma}{Nov 23, 2020}{35}{Extremal Set Theory}{Mohit Singla}{$\alpha$}{JS}
\section{Recall Sperner Theorem}
\begin{theorem}
Let $F$ be the family of subsets of $[n]$ $|$ $\forall$ $A,B \in F$, $A \nsubseteq B$.
$$|F| \le {n\choose{n/2}}$$
The size is the width of subset boolean poset.
\begin{proof}
\textbf{Tightness:-} The bound is tight as we have example $F=\{A\subseteq [n] \mid \left |A\right |=n/2\}$\\\\
 A permutation $\pi \in S_n$ is said to meet $A \subseteq \{1,2,\ldots,n\}$ if $A$ forms prefix of $\pi$. Let's say $|A| = k$ then $\pi$ said to meet $A$ if $A = \{\pi(1),\pi(2),\ldots,\pi(k)\}$.\\\\
Consider each subset in $F$ and consider permutations meeting them, As we are taking subsets from $F$, they are incomparable. Hence a single permutation can not meet both $A$ and $B$. So,
$$\sum_{A\in F} \bigg|\{\pi | \pi ~meets~ A\}\bigg| \leq n!$$\\
Now number of permutations that can meet set $A$ of size $k$ is $k! \times (n-k)!$.So,
$$\sum_{A\in F} |A|! \times (n-|A|)! ~~~\leq~~~ n!$$
$$\boxed{\sum_{A\in F} \frac{1}{{n \choose {|A|}}} ~~~ \leq ~~~ 1}$$\\
The above inequality is known as \textbf{LYM Inequality -  Lubell–Yamamoto–Meshalkin Inequality,}\\\\
We can substitute $n \choose \frac{n}{2}$ in place of $|A|$ and the inequality still holds.
$$\sum_{A\in F} \frac{1}{{n \choose \frac{n}{2}}} ~~~ \leq ~~~ 1$$
$$\sum_{A\in F} 1  ~~~ \leq ~~~ {n \choose \frac{n}{2}}$$
$$|F| ~~~\leq~~~ {n \choose \frac{n}{2}}$$\\
\end{proof}
\end{theorem}
\section{Disussing family of subsets with certain intersection properties}
The \textbf{Sperner theorem} we did is sample of family of subsets with intersection properties : for any pair $A,B \subseteq F ~|~ A\cap B \neq A$\\\\
\textbf{Question:-} Consider  family of subsets of $[n]$ such that for any pair $A,B \subseteq [n]$, $A \cap B \neq \phi$. How large can $F$ be?\\\\
\textbf{Building up:-} The size can be as large a $2^n$ but will not be equal to $2^n$ as if we take all singleton sets, their pairwise intersection is empty.\\ The size can be $2^{n-1}$, $F = \{ B \cup \{n\}~ | ~B\subseteq[n-1] \}$. Example for $n=3$ : $F=\{\{3\},\{1,3\},\{2,3\},\{1,2,3\}\}$. Can it be larger than $2^{n-1}$ ?
\begin{theorem} The size of intersection family is atmost $2^{n-1}$. And this bound is tight as we already have an example for any $n$.
\begin{proof}
By the property of family of subsets, its clear that if any subset $A \in F \implies {A}^\complement \notin F$. If both $A$ and $A^\complement$ are present then their intersection is empty. So for every subset in $F$ there is subset not present in $F$. Hence $|F| \le 2^{n-1}$.
\end{proof}
\end{theorem}
\textbf{Special cases of above family:-}
\begin{enumerate}
\item The intersection size is always $\lambda$ i.e. $A,B \subseteq F ~ \mid ~ |A \cap B|=\lambda$. Claim : $|F|\le n$.
\item The family is $k$-uniform.
\end{enumerate}
We will answer above two cases using \textbf{Linear Algebra} techniques. Before proving above claims lets take a look at some other problem that uses similar proof techniques.

\subsection{Odd Town Problem}
\begin{itemize}
    \item $n$ people in an odd town form $m$ clubs $C_1, C_2, \hdots, C_m$.
    \item Each club has an odd number of members.
    \item Each pair of clubs have an even number of common members.
    \item No two clubs have same set of members.
\end{itemize}
\begin{theorem}In Odd Town problem $m \le n$.
\begin{proof}
Associate a $n$-sized 0-1 vector $v_i$ to each club $C_i$, such that $v_i[j]=1$ if $j$ is member of club $C_i$ else 0. It is now sufficient to prove that we cannot have more than $n$ different vectors. Theses vectors are defined over field $\mathbb{F}_2^n$. The field $\mathbb{F}_2^n$ can be considered as structure where both addition and multiplication is modulo 2 and every element is either 0 or 1.\\\\
If we can show that the set of vectors $v_1, v_2, \hdots, v_m$ are linearly independent over $\mathbb{F}_2^n$, then it's implied that $m \le n$ as number of independent vectors are always less than or equal to dimension.\\\\
Consider the inner product $\langle v_i,v_j\rangle$. In $\mathbb{F}_2^n$, $\langle a,b\rangle = (\sum_{1}^{n} a_ib_i)\%2$.\\

~\textbf{Case 1:} $i\neq j$, $\langle v_i,v_j\rangle = \sum_{k=1}^{n} v_i[k]v_j[k] = | C_i \cap C_j |\%2 = 0$. As the size of intersection is always $even$.\\

~\textbf{Case 2:} $i= j$, $\langle v_i,v_j\rangle = \sum_{k=1}^{n} v_i[k]v_i[k] = (\sum_{k=1}^{n} v_i[k])\%2= 1$. As the size of each club is $odd$.\\\\
Suppose the set of vectors $v_1, v_2, \hdots, v_m$ are not linearly independent. Then these exist non-trivial solution to the equation: $\sum \lambda_i v_i=0$. If we can show that only solution to above equation is when all $\lambda_i$ is \textbf{0}, then we are done with proof.\\\\
Take inner product of $\sum \lambda_i v_i$ with $v_j$ for every $j$. As $\sum \lambda_i v_i$ is \textbf{0} vector.
$$ \left \langle \sum \lambda_i v_i,v_j\right\rangle = 0$$
$$  \sum \left \langle \lambda_i v_i,v_j\right\rangle = 0$$
$$  \sum \lambda_i \left \langle  v_i,v_j\right\rangle = 0$$\\
As we saw earlier, when $i\neq j$ $\langle v_i,v_j\rangle=0$. And when $i= j$, $\langle v_i,v_j\rangle=1$. So the above equation can be written as:
$$  \lambda_j \left \langle  v_j,v_j\right\rangle = 0$$
For above equation to satisfy $\lambda_j=0$, this can be shown for all $\lambda$s. Hence proved $m\le n$.
\end{proof}
\end{theorem}
In next lecture we will analyse the special cases of intersection family.


\Lecture{Jayalal Sarma}{Nov 25, 2020}{36}{More on Linear Algebra Techniques}{Mohit Singla}{$\alpha$}{JS}
\section{Recall some intersection families}
The family $F$ such that for any two subsets $A,B\subseteq F ~|~ A\cap B \neq \phi$, then $|F| \le 2^{n-1}$. Then we questioned two special cases of this intersection family.
\begin{enumerate}
    \item The intersection size is always $\lambda$ i.e. $A,B \subseteq F ~ \mid ~ |A \cap B|=\lambda$. Claim : $\boxed{|F|\le n}$. This inequality is knows as \textbf{Fisher's Inequality}.
    \item The family is $k$-uniform $\lambda$ i.e. $A \subseteq F ~ \mid ~ |A|=k$. Claim : $\boxed{|F|\le {{n-1}\choose{k-1}}}$. This is knows as \textbf{Edr{\"o}s Ko-Rado Theorem}.
\end{enumerate}
We saw \textbf{Odd-Town Theorem} which states that if:
\begin{itemize}
    \item $n$ people in an odd town form $m$ clubs $C_1, C_2, \hdots, C_m$.
    \item Each club has an odd number of members.
    \item Each pair of clubs have an even number of common members.
    \item No two clubs have same set of members.
\end{itemize}
Then $m\le n$. We proved this inequality using \textbf{Linear Algebra}. We associated one vector with each club and showed that all the vectors are linearly independent. As the dimension is $n$, there can not be more that $n$ linearly independent vectors.

\section{Fisher's Inequality (1940s)}
We will look at proof by \textbf{Babai Frankl} in 1992.
\begin{theorem}
The family $F$ such that intersection size is always $\lambda$ i.e. $A,B \subseteq F ~ \mid ~ |A \cap B|=k $. then $|F|\le n$.
\begin{proof}Associate a $n$-sized 0-1 vector $v_i$ to each club $C_i$, such that $v_i[j]=1$ if $j$ is member of club $C_i$ else 0. Each $v_i\in \mathbb{R}^n$.\\\\
Consider the inner product $\langle v_i,v_j\rangle$. In $\mathbb{R}^n$, $\langle a,b\rangle = (\sum_{1}^{n} a_ib_i)$.\\

~\textbf{Case 1:} $i\neq j$, $\langle v_i,v_j\rangle = \sum_{k=1}^{n} v_i[k]v_j[k] = | C_i \cap C_j | = k$. As the size of intersection is always k.\\

~\textbf{Case 2:} $i= j$, $\langle v_i,v_j\rangle = \sum_{k=1}^{n} v_i[k]v_i[k] = \sum_{k=1}^{n} v_i[k]= |C_i|$.\\\\
We will show the vectors are linearly independent. Suppose the set of vectors $v_1, v_2, \hdots, v_m$ are not linearly independent. Then these exist non-trivial solution to the equation: $\sum \lambda_i v_i=\textbf{0}$. We can write\\
$$ \left \langle \sum_1^m \lambda_i v_i,\sum_1^m \lambda_i v_i \right\rangle = 0$$
$$\implies ~~ \sum_1^m \lambda_i^2 \left \langle   v_i, v_i \right\rangle + \sum_{1\le i \neq j\le m} \lambda_i \lambda_j \left \langle   v_i, v_j \right\rangle = 0$$
$$\implies ~~ \sum_1^m \lambda_i^2 |C_i| + \sum_{1\le i \neq j\le m} \lambda_i \lambda_j k = 0$$
$$\implies ~~ \sum_1^m \lambda_i^2 (|C_i|-k) +\sum_1^m \lambda_i^2k + \sum_{1\le i \neq j\le m} \lambda_i \lambda_j k = 0$$
$$\implies ~~ \sum_1^m \lambda_i^2 (|C_i|-k) +(\sum_1^m \lambda_i)^2k = 0$$\\\\

Suppose there exist a $C_i$ such that $|C_i|=k$, then no other club size can be $k$. If there are two clubs with size $k$ and their intersection is also of size $k$, then both the clubs have to be equal. so at-most one club can have size $k$. Now as the intersection size is always $k$, for all $j$, $C_i \subseteq C_j$ must hold as the size of $C_i$ is $k$ and $|C_i \cap C_j |=k$. Note for all $C_i$, $|C_i|\ge k$ as the intersection size with any other club is of size $k$.\\\\
So in the expression $\sum_1^m \lambda_i^2 (|C_i|-k) +(\sum_1^m \lambda_i)^2k$, both the parts in summation are non-negative. For right part to be zero, $\sum_1^m \lambda_i =0$. Not all $\lambda$s are zero but their summation is, so there are atleast 2 $\lambda$s which are non-zero.\\\\
Now lets focus on left part. Each term $\lambda_i^2 (|C_i|-k)$ is non-negative.  There are atleast 2 $\lambda$s which are non-zero but atmost one $(|C_i|-k)$ can be zero. So the summation is always positive. So if there are non-zero $\lambda$s then summation cannot be equated to zero. This leads to contradiction in assumption that there exists non-zero $\lambda$s which satisfy above equation.\\\\
Hence proved that vectors $v_1, v_2, \hdots, v_m$ are linearly independent. So $m\le n$.
\end{proof}
\end{theorem}

\section{Application of Fisher Inequality and Odd Town Theorem}
\subsection{Ramsey Number}
\textbf{Definition:}$R(s,t)$ is minimum number ($n$) of vertices required in complete graph such that 2-edge coloring (red and blue colors) of this $K_n$ produces either red $K_s$ or blue $K_t$.\\\\
Earlier we have proved $R(t,t)>2^t$ i.e. there exist a 2-edge coloring of $K_{2^t}$ such that there is no red $K_t$ or blue $K_t$. We proved this using probabilistic method. We chose color of each edge uniformly at random and observed that the probability of graph having either red $K_t$ or blue $K_t$ is strictly less than 1. So there exist a graph with neither has red $K_t$ nor blue $K_t$, we did not explicitly drew thew the graph. This proof was \textbf{non-constructive}.\\\\
If we want a constructive example for lower-bound, we have much weaker lower bound.
\subsection{Constructive lower bound for diagonal Ramsey number}
\begin{claim}
$R(t+1,t+1)> {t\choose 3}$\\\\
We will be able to show a construction to prove this claim.
\begin{proof}
We need to show there exist a 2-edge coloring of $K_{t\choose 3}$ such that there is no red $K_{t+1}$ or blue $K_{t+1}$.\\\\
$n={t\choose 3}$. Interpret each vertex as 3-sized subset of set $\{1, 2, \hdots, t\}$. Let $A,B \in V$, then $A,B \subseteq [t]$ and $|A|=|B|=3$. Color edge $AB$ red if $|A\cap B| =0$ or $2$. As size of each subset is 3 and all are pairwise distinct, the intersection size can only be 0, 1 or 2. Color edge $AB$ blue if $|A\cap B| =1$.\\\\
Lets look for blue $K_{t+1}$. Consider $F$ as family consisting 3-sized subsets of $[t]$ such that intersection size is 1. These subsets in $F$ will represent the vertices corresponding to blue edges as described earlier. By Fishers theorem we know $|F|\le t$. There are atmost $t$ vertices available so there cannot be $K_{t+1}$.\\\\
Lets look for red $K_{t+1}$. Consider $F$ as family consisting 3-sized subsets of $[t]$ such that intersection size is even. These subsets in $F$ will represent the vertices corresponding to red edges as described earlier. Now we have all subsets of odd size and intersection size even. All conditions required in Odd Town theorem are satisfied, so by the theorem $|F|\le t$. There are atmost $t$ vertices available so there cannot be $K_{t+1}$.\\\\
Hence proved.
\end{proof}
\end{claim}

\section{Edr{\"o}s Ko-Rado Theorem (Discovered-1938, Presented-1962)}
\begin{lemma} Let $C$ be a cycle of length $n$ ($n$ edges and $n$ vertices). Let $H$ be family of paths in $C$ of fixed length $k$ where $k\le \frac{n}{2}$. Assume any paths in $H$ have an common edge. Then $|H|\le k$.
\begin{proof}Pick a path $p = (v_1, v_2, \hdots v_{k+1})$. All other paths have to intersect with this path. A path here is contiguous set of edges in cycle. Lets analyse how other paths look like. No other path can start from $v_1$, as it will end up being the same path $p$. No path can start at $v_{k+1}$, as $k\le \frac{n}{2}$ and path starting at $v_{k+1}$ will not have any edge common with $p$. Similarly no path can end at $v_1$ and $v_{k+1}$. So paths can start or end at $v_2, v_3 \hdots v_{k}$. Note if there is a path that starts at $v_j$, we cannot include path ending at $v_j$ as these two paths will not have any common edge. So there can at-most be $k-1$ other paths. Hence $|H|\le k$.
\end{proof}
\end{lemma}

\begin{theorem}[{\bf Erd{\"o}s-Ko-Rado Theorem}]
$F$ is $k$-uniform where $k\le \frac{n}{2}$, family of subsets of $[n]$ such that for every $A,B \subseteq F$, $A\cap B \neq \phi$. Note if $k> \frac{n}{2}$, then every pair of subsets trivially has non-empty intersection. Then $|F|\le {{n-1}\choose{k-1}}$

\begin{proof}
Before we prove the theorem, to get a feel, we construct an example first. And this example also will prove that the above theorem is tight.

\paragraph{Tightness:} 
Consider the family.
$$F_k=\{\{n\}\cup B ~|~ B\subseteq [n-1] ,|B|=k-1\}$$

Since $n$ is there in every set in the family, the family is an intersecting family. All sets have size $k$ and hence it is $k$-uniform. By definition, $|F_k|={{n-1}\choose{k-1}}$. Thus the claim is tight as we have an example for any $k$. It turns out that these are the only tight examples when $k < \frac{n}{2}$.

When $n$ is even and $k=n/2$, there is one more tight example. We can take $F_k$ such that for every $\frac{n}{2}$-sized subset of $[n]$, we take either that subset or its complement in $F$. 
$$|F_k|={{{n}\choose{n/2}}* \frac{1}{2}}=\frac{n}{n/2}* {{n-1}\choose n/2-1}* \frac{1}{2} = {{n-1}\choose{n/2-1}}$$

\paragraph{Proof of the {\bf Erd{\"o}s-Ko-Rado Theorem}:} We will discuss the proof by \textbf{Katona (1972)} using cycle permutation argument. 

Assume someone invited all of the $n$ people from the town to a party. These $n$ people are members of clubs $C_1, C_2, \hdots, C_m$ where each club is of size $k$ and every pair of intersection is non-empty. The party has round table with $n$ labelled chairs. Host wants to seat the club members contiguously. But of course he may not be able to since there may be contradictory requirements across clubs since members can be common. 

However, in an attempt to maximise the seating satisfaction of the club members, he decided to try all $n!$ permutations. Lets define a Matrix $H$ with $n!$ rows indexing the permutation number which represents seating arrangement. The columns are indexed by club $C_1, C_2, \hdots, C_m$, so total $m$ columns. The entries of the matrix defined as follows. For $\sigma \in S_n, j \in [m]$:

$$H(\sigma,j) = 
\begin{cases}
1 & \textrm{~if $\sigma$ seats members of club $C_j$ contiguously} \\
0 & \textrm{otherwise} \\
\end{cases}
$$

Let $c$ be the number of ones in matrix $H$. we will count $c$ in two different ways.

\begin{description}
\item{\textbf{Count 1 - Row-wise first:}} For a given permutation, each club seated contiguously is a path in the cycle. The guests correspond to edges. As the family of clubs is an intersecting family, by above lemma we know that in each row there can be atmost $k$ ones. so $c \le k*n!$
\item{\textbf{Count 2 : Column-wise first:}} As we are considering all the permutations possible, the number of times each club appears contiguously will be the same. In fact a club will appear contiguously in $nk!(n-k)!$ permutations. First choose the starting index from n positions, then permute the members inside group, and then permute the members outside group. There are $m$ columns, so $c=mnk!(n-k)!$.
\end{description}
Using the above two counts, 
\begin{eqnarray*}
m~n~k!~(n-k)! & \le & k~n! \\ 
m & \le & \frac{kn!}{nk!(n-k)!} \le \frac{(n-1)!}{(k-1)!(n-k)!} \le {n-1 \choose k-1}
\end{eqnarray*}


\end{proof}
\end{theorem}

\Lecture{Jayalal Sarma}{Nov 27, 2020}{37}{Generalization of linear algebraic method }{Banavath Tarun}{$\alpha$}{JS}

\section {Running problems}

\begin{enumerate}
    \item \underline{\textbf{Intersecting family}} : Let $\mathcal{F}$ be family of subsets of $\{1,2,3,...,n\}$ and is called Intersecting family, if every two subsets in these family intersect (i.e. $\forall A, B \in \mathcal{F}$, ~~$A \cap B \ne \phi$), and we have seen that $|\mathcal{F}|\le 2^{n-1}$.
    
    \item \underline{\textbf{Intersecting family with fixed size $\lambda$ (Fisher's Inequality)}} : Similarly, when $\mathcal{F}$ is a family of subsets of $\{1,2,3,....,n\}$ such that $\forall A, B \in \mathcal{F}$, $~~~|A\cap B| = \lambda$, then size of family is not too large and is given as $|\mathcal{F}|\le n$
    \subitem \textbf{techniques used in Fisher's inequality :} Odd town problem, linear algebra method. 
\end{enumerate}

In this lecture, we discuss the theorems involving polynomial methods (or) function space methods.
\\

\underline{proof techniques for polynomial method (or) function space methods includes 3 steps:}

\begin{enumerate}
     \item Associate a polynomial in $n$-variables with each element 
     \item prove that polynomials are linearly independent
     \item Bound the dimensions of space of polynomials
\end{enumerate}




\section{Independence criterion Tool and Two-distance set}

\begin{lemma}
(\textbf{Independence criterion Tool}) $\forall ~1 \le i \le m$, let $f_i : \Omega \longrightarrow \mathbb{F}$ be a function, $v_i \in \Omega$ such that it satisfies the following two conditions : 
\begin{enumerate}
    \item $f_i(v_i) \ne 0, ~\forall ~i$
    \item $f_i(v_j) = 0, ~\forall ~1 \le i<j \le m $
\end{enumerate}
then $\{f_1, f_2, f_3,....,f_m\}$ are linearly independent in $\mathbb{F}^{\Omega}$. (Here each function $f_i$ is a element in $\mathbb{F}^{\Omega}$ i.e. a vector)
\begin{proof}
Suppose $\exists \lambda_1, \lambda_2, \lambda_3,...,\lambda_m \in \mathbb{F}$ such that $F = \sum_{i =1}^{m} \lambda_i f_i = 0$ and suppose there are dependent, where not all $\lambda_i = 0$, then there is a contradiction. 
Let $j$ be the largest index (rightmost) such that $\lambda_j \ne 0.$ Substitute $v_j$ to the above function $F$.
We then have $F(v_j) = \sum_{i=1}^{m} \lambda_i f_i(v_j) = \lambda_j f_j(v_j)$ (because $\forall i > j $ $\lambda_i = 0$ by choice of $j$ and $\forall i< j$ the term $f_i(v_j) = 0$ by second condition) then $\lambda_j f_j(v_j) = 0$. Since $f_j(v_j) \ne 0$, $\lambda_j$ should be zero, but $\lambda_j \ne 0$ from above assumption. Thus it contradicts the fact that $\exists~ \lambda_1, \lambda_2, \lambda_3,...,\lambda_m \in \mathbb{F}$ that are dependent. 
\end{proof}
\end{lemma}

\begin{claim}
Let $a_1,a_2,....,a_m$ be $m$ points in $\mathbb{R}^{n}$, such that all pair-wise distances are unique, them relation between $m$ and $n$ is given as : $m \le n+1$ 
\end{claim}
What if we relax the condition in above claim and require that there are two possible distances c, d, so that any pairwise distance is either c or d? Such a set is called a two-distance set.
\\
Below theorem is an example of demonstration of a polynomial method (or) function space method.
\\
\begin{theorem}(\textbf{Two-distance set}) Consider a two-distance set, a set $S = \{a_1,a_2,a_3,...,a_m\}$ is said to two-distance set, if $\exists$ $d_1, d_2$ (two fixed distances) such that $\forall a_i, a_j \in S$, satisfy the relation $dist(a_i, a_j) = d_1$ or $dist(a_i, a_j) = d_2$ (where $dist(a_i, a_j)$ is the distance between the two points $a_i,a_j$ in $\mathbb{R}^{n}$).
Then the relation between $m$ (size of two-distance set) and $n$ (dimension of elements in two-distance set) is given as : $m \le {n \choose 2} + 3n +2$.
\end{theorem}
\begin{proof}
We Solve it using polynomial method (or) function space methods that follows the three steps as already mentioned.
\\

\underline{\textbf{step1 (Associate a polynomial)}} : 
Associate a polynomial $P_i(x_1,x_2,x_3,....,x_n)$ for each element $a_i$ in $S=\{a_1,a_2,a_3,....,a_m\}$ with some properties, so that we can prove independence in these polynomials. As seen in above lemma the functions $f_i$'s are polynomial $P_i$'s here and $\Omega = \mathbb{R}^{n}$ in this case (since $\forall i,~a_i \in \mathbb{R}^{n}$). Let $x=(x_1,x_2,x_3,...,x_n)$ be a $n$-vector then polynomial : 
$\forall ~1 \le i \le m ~P_i(x) = \big({||x-a_i||}^2-{d_1}^2\big)\big({||x-a_i||}^2-{d_2}^2\big)$.
\\

\underline{\textbf{step2 (prove that polynomials are linearly independent)}} : 
The above polynomials defined satisfies the two conditions of independent criterion tool. i.e.
\begin{enumerate}
    \item $P_i(a_i) \ne 0,~ \forall i$. $P_i$ satisfy the first condition because $P_i(a_i)= \big({||a_i-a_i||}^2-{d_1}^2\big)\big({||a_i-a_i||}^2-{d_2}^2\big) = {{d_1}^2}{{d_2}^2} \ne 0$ (since $d_1,d_2 \ne 0$), So  $P_i(a_i) \ne 0,~ \forall i$
    \item $P_i(a_j) = 0, \forall 1 \le i <j \le m$. $P_i$ also satisfy the second condition because $P_i(a_j) = \big({||a_j-a_i||}^2-{d_1}^2\big)\big({||a_j-a_i||}^2-{d_2}^2\big)$ and also ${||a_j - a_i||}^2 = {{d_1}^2}$ (or) ${||a_j - a_i||}^2 = {{d_2}^2}$ from definition of two-distance set. So $P_i(a_j) = \big({||a_j-a_i||}^2-{d_1}^2\big)\big({||a_j-a_i||}^2-{d_2}^2\big) = 0, \forall 1 \le i <j \le m$
\end{enumerate}


\underline{\textbf{step3 (Bound the dimension of space of polynomials)}} :  
The polynomial that we have constructed has a property that we wanted and all these polynomials are now independent by independent criteria. Also after expanding the polynomial $P_i(x=(x_1,x_2,...x_n)) = \big({||x-a_i||}^2-{d_1}^2\big)\big({||x-a_i||}^2-{d_2}^2\big)$, we can see that the polynomial have only following type of terms:
$$(\sum_{i=1}^{n}{x_i}^{2})^2, {\sum_{i}{x_i}^{2}} x_j, {\sum_{1 \le i\le j \le n}x_ix_j}, \sum_{i=1}^{n}x_i ,1$$.
i.e. 
\begin{itemize}
    \item Number of terms of form $(\sum_{i=1}^{n}{x_i}^{2})^2$ are $1$. 
    \item Number of terms of form ${\sum_{i}{x_i}^{2}} x_j$ are $n$. 
    \item Number of terms of form ${\sum_{1 \le i\le j \le n}x_ix_j}$ are ${n \choose 2} + n$. 
    \item Number of terms of form $\sum_{i=1}^{n}x_i$ are $n$. 
    \item Number of constant terms are $1$. 
\end{itemize}
So in total we have $1+n+{n \choose 2} + n + n +1  = {n \choose 2} + 3n + 2$ number of terms i.e. any polynomial in our family can be expressed as a linear combination in ${n \choose 2} + 3n + 2$ number of terms, which means that dimensions of set of polynomials that we are looking cannot be greater than ${n \choose 2} + 3n + 2$ because these are the simplified polynomials using which we can express all the polynomials in our family. So, the dimension of underline space is bounded by ${n \choose 2} + 3n + 2$ and hence 
$$m \le {n \choose 2} + 3n + 2$$ because there cannot be more than ${n \choose 2} + 3n + 2$ number of independent polynomials (or) vectors.
\end{proof}

\section{Frankl-Wilson Theorem}
\begin{theorem}
Let $\mathcal{F}$ be family of subsets of $[n]$ and let $L$ be subset of $[n]$ ($L \subseteq \{1,2,3,...,n\}$) such that $\forall A,B \in \mathcal{F}$,  $~~|A\cap B| \in L$, then $|\mathcal{F}| \le \sum_{i=1}^{|L|} {n \choose i}$. This theorem is also known as \textbf{Frankl-Wilson Theorem}, which is a Generalization of \textbf{Fisher's theorem}.
\end{theorem}

\begin{proof}
Let $\mathcal{F} = \{A_1,A_2,....,A_m\}$ be family of subsets of $[n]$, $L={l_1,l_2,....,l_s}$ be a subset of $[n]$ such that $\forall i,j ~(1 \le i,j \le n)$ $\exists~ k~(1 \le k \le s)$ such that $|A_i\cap A_j| = l_k$. Let $v_i$ be the characteristic vector of $A_i$, $v_i \in \Omega = \mathbb{R}^n$ \big(where $<v_i,v_i> = |A_i|$ and $<v_i,v_j> = |A_i\cap A_j|\big)$ and we can assume that $A_1$ to $A_m$ is ordered such that sizes are non-decreasing order (or) rename them accordingly. We then have
$$|A_1| \le |A_2|  \le |A_3|..... \le |A_m|.$$ 
Now We Solve the proof using polynomial method (or) function space methods that follows the three steps as already mentioned.
\\

\underline{\textbf{step1 (Associate a polynomial)}} : 
Associate a polynomial $P_i(x_1,x_2,....,x_n)$ for each element $A_i$. So the polynomial that we are going to define is as follows:
$$P_i\big(x = (x_1,x_2,....,x_n)\big) = \prod_{k ~: ~l_k < |A_i|}\big(<v_i,x>-l_k\big)$$
where $x=(x_1,x_2,....,x_n)$ is a $n$-vector.
\\


\underline{\textbf{step2 (prove that polynomials are linearly independent}} : The above polynomials defined satisfies the two conditions of independent criterion tool.
\begin{enumerate}
    \item 
    when $x=v_i$, we have:
$$P_i(v_i) = \prod_{k ~: ~l_k < |A_i|}\big(<v_i,v_i>-l_k\big) = \prod_{k:l_k < |A_i|}\bigg(|A_i|-l_k\bigg) \ne 0$$
so, $P_i(v_i) \ne 0,~ \forall i$
    \item
    when $x=v_j$ and $i < j$, we have:
$$P_i(v_j) = \prod_{k ~: ~l_k < |A_i|}\big(<v_i,v_j>-l_k\big) = \prod_{k:l_k < |A_i|}\big(|A_i\cap A_j|-l_k\big)=0$$
Since $|A_i \cap A_j| $ must be one of the $l_p \in L$ and this $p$ must satisfy $l_p < |A_i|$, because the intersection size $|A_i \cap A_j|$ cannot be more than $A_i$.
So, $P_i(v_j) = 0, \forall 1 \le i <j \le m$
Now we have this polynomial $P_i\big(x = (x_1,x_2,....,x_n)\big) = \prod_{k ~: ~l_k < |A_i|}\big(<v_i,x>-l_k\big)$ that satisfies the two conditions of Independent criteria and hence $P_1,P_2,....,P_n$ are linearly independent as polynomials (or) as coefficients of vectors.
\end{enumerate}

\underline{\textbf{step3 (Bound the dimension of space of polynomials)}} : 
Estimating the underlying dimensions (here, space of polynomials):
We need to compute the dimension of space containing all these polynomials. Since we are using only 0/1 vectors here we can reduce the dimension by replacing the higher powers ${x_i}^k$ with $x_i$, this process does not change the linear dependence property and we still get same conditions. Now each term monomial looks like $x_1x_2x_3....$. Polynomials with these properties (i.e. each monomial with every individual variable degree to be at most one) are called as multilinear polynomials. So the polynomials under consideration lives in the space of multilinear polynomials of degree at most $s$.
\\
Let us look at the example on multilinear polynomials when $n=3$ and degree=2, the only terms correspond to these multilinear polynomials looks like:
$$x_1 x_2, ~x_2 x_3, ~x_1 x_3, ~x_1, ~x_2, ~x_3, ~1$$ there are 7 terms and an example of these multilinear polynomial with real coefficients looks like:
$5x_1 x_2 + 6x_1 x_3 + 5x_2 x_3 + 3x_1 + 2x_2 + x_3 + 6$
and dimension of this multilinear polynomial example is $7$. The number of monomials we can have with that much degree is the bound for dimension. So in general when degree=$k$ and number of variables is $n$, then number of monomials in this multilinear polynomial is $n \choose k$ (i.e. select $k$ elements from $\{1,2,3,....,n\}$ and associate a monomial with corresponding variables). But now we have that degree is at most $s$, so we need to sum the result $n \choose k$ overall values of $k$ (i.e. $1 \le k \le s$). So,
$$Dimension ~of ~space = \# ~of~ subsets~ of ~\{1,2,3,...,n\} ~of~ size ~\le s = \sum_{k=1}^{s}{n\choose k}$$
and hence
$$m \le Dimension \le \sum_{k=1}^{|L|}{n \choose k}$$
\end{proof}


\begin{theorem}
Let $\mathcal{F}$ be family of subsets of $[n]$, $p$ be a prime and $L$ be subset of $\{0,1,2,3,...,p-1\}$, ($L \in \{0,1,2,3,...,p-1\}$) such that $\forall A, B \in F$, $~~|A \cap B| \in L (mod~p)$ and $|A| \notin L (mod~p)$, then $|\mathcal{F}| \le \sum_{i=1}^{|L|} {n \choose i}$. This theorem is proved by \textbf{Ray-choudhuri Wilson}, which is a Generalization of \textbf{odd-town problem}. 
\end{theorem}

\begin{theorem}
Let $\mathcal{F}$ be a k-Uniform family of subsets of $[n]$ and let $L$ be subset of $[n]$,  ($L \subseteq \{1,2,3,...,n\}$) such that $\mathcal{F}$ is $L-$intersecting (i.e. $\forall A, B \in F$, $~~|A \cap B| \in L$), then the size of $\mathcal{F}$ is given as : $|\mathcal{F}|\le {n \choose |L|}$. This theorem is proved by \textbf{Ray-choudhuri Wilson}  
\end{theorem}

The above two theorems are also applications of polynomial method and process involved in proving them is similar to that we have done before.
